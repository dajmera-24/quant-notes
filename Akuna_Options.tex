\documentclass{scrartcl}
\usepackage{amsmath, amssymb, amsthm}
\usepackage{parskip}
\usepackage{graphicx, xcolor}
\usepackage{gensymb}
\usepackage[most]{tcolorbox}

\tcbset{
  theorem style/.style={
    enhanced,
    breakable,
    colback=#1!5,      % background
    colframe=#1!60,    % border
    boxrule=0.5pt,
    arc=2pt,
    left=6pt,right=6pt,top=4pt,bottom=4pt
  }
}

\definecolor{thmblue}{HTML}{3A6EA5}      % calm academic blue
\definecolor{defpurple}{HTML}{6B4E9B}    % definition = conceptual
\definecolor{examber}{HTML}{B07C2C}      % examples = warm + applied
\definecolor{probgray}{HTML}{555555}     % neutral problems
\definecolor{remolive}{HTML}{6A7F2A}     % remarks = side info
\definecolor{solteal}{HTML}{2F8F83}      % solutions = resolution
\definecolor{propindigo}{HTML}{3F4C81}   % propositions = formal

\newtcbtheorem[number within=section]{theorem}{Theorem}%
  {theorem style=thmblue}{thm}

\newtcbtheorem[number within=section]{definition}{Definition}%
  {theorem style=defpurple}{def}

\newtcbtheorem[number within=section]{example}{Example}%
  {theorem style=examber}{ex}

\newtcbtheorem{problem}{Problem}%
  {theorem style=probgray}{prob}

\newtcbtheorem[number within=section]{remark}{Remark}%
  {theorem style=remolive}{remk}

\newtcbtheorem[number within=section]{solution}{Solution}%
  {theorem style=solteal}{sol}

\newtcbtheorem[number within=section]{proposition}{Proposition}%
  {theorem style=propindigo}{prop}

\title{Akuna Options 101 Notes}
\author{Dhruv Ajmera}
\date{}

\begin{document}

\maketitle

\tableofcontents

\clearpage

\section{Terminology}

We provide some basic terminology underlying options.

\subsection{Basics}

\begin{definition}{Bid}
.The Bid is the (highest) price for which someone is willing to buy something.
\end{definition}

\begin{definition}{Offer}
.The Offer is the (lowest) price at which someone is willing to sell something.
\end{definition}

\begin{definition}{Size}
.The Size is the amount of contracts one is willing to trade at a price.
\end{definition}

\begin{remark}{Cushion}
.One tends to include a 'cushion' between the expected value and bid/offer.
\end{remark}

The idea of `market making' is thus providing bid, ask and sizes for each.

\begin{example}{Temperature Betting}
.The expected temperature tomorrow is $64 \degree$. If one were to offer 4 bid contracts at $60 \degree$ and 10 offer contracts at $68 \degree$, then they are `60 bid for 4 and have 10 at 68'; `60 at 68, 4 by 10'.
\end{example}

\begin{definition}{Spread}
.The Spread is the difference between the bid and ask price.
\end{definition}

\begin{definition}{Hedge}
.A trade or investment to reduce the risk of price movement in an asset (e.g. if we bet on a team to win, we `hedge' this bet by making secondary bets against points, halftime score, 3 point \%, etc.).
\end{definition}

\begin{definition}{Paper}
.Interested parties trading against the market makers.
\end{definition}

\begin{definition}{Broker}
.An intermediary between buyers and sellers.
\end{definition}

\begin{definition}{Tick Size}
.The increment between one level and the next level (Stock goes from 100.00 \textrightarrow 100.1 indicates tick size of 0.01).
\end{definition}

\begin{definition}{Queue Priority}
.A structure used to determine the right of precedence within those in a list (commonly Price-time priority \textrightarrow separate orders into price bands, and then organize by time of request with highest bidders on top).
\end{definition}

\begin{definition}{Spread}
.The Spread is the difference between the bid and ask price.
\end{definition}

\subsection{Order Types}

\begin{definition}{Immediate or Cancel}
.(IOC) A type of order that requires all or part of the order to be executed immediately. Unfulfilled parts of order are cancelled.
\end{definition}

For example, if we have 100 bid for 24 contracts, and an IOC order to sell 50 at 100 is recieved \textrightarrow we sold 24 at 100 and the remaining 26 are cancelled (can only exec. at any price at or above your offer, for a qty. upto your size)

\begin{definition}{Good for Day}
.(GFD) This order remains until executed or until end of trading day.
\end{definition}

\begin{definition}{Good-til-cancelled}
.(GTC) This order remains until cancelled or completed.
\end{definition}

\begin{definition}{All or None}
.(AON) Must be executed in its entirety, or not executed at all (more uncommon).
\end{definition}

\begin{definition}{Fill or Kill}
.(FOK) Must be executed immediately in its entirety, otherwise cancelling (few seconds in floor trading).
\end{definition}

Note that Fill and Kill (FAK) is synonymous to IOC.

\begin{definition}{One cancels the other}
.(OCO) When one order is executed, the other is cancelled (used to prevent over-exposure).
\end{definition}

\begin{definition}{Contract Size}
.The multiplier attached to an option or future. Options on stock generally have a multiplier of 100 shares. Options on futures have a multiplier of 1 future. The multiplier on options on a future and the multiplier on the future can vary.
\end{definition}


\begin{definition}{Theoretical Value}
.(Theo) based on all inputs, the current value a market maker believes an option is worth.
\end{definition}

\begin{definition}{Sheets}
.(Fair Value) same as above, but generally when referring to where something traded.
\end{definition}

\begin{definition}{Liquidity}
.(Fair Value) how easy/hard it is to trade close to fair value. Generally determined by the number on contracts on the bid/offer, along with the width of the market.
\end{definition}

\section{Profit Mechanisms}

One primary profit mechanism is pocketing the bid-ask spread. The market maker calculates the value for an option and \textbf{disseminates} a bid below this value and an offer above this value to the market.

But how can one sell an asset they do not own?

An options contract is created when two parties agree to interact \textrightarrow it is a promise to deliver what is being optioned at a later date. There are a few outcomes when the buyer exercises the option:

\begin{itemize}
  \item Stock Option: deliver (buy) the underlying shares of stock 
  \item Cash Settled Option: debit the cash from ones account to pay the buyer for the difference between the transaction price and the settlement price
  \item Future Settled Option: deliver a \textbf{future} (underlying contract created when traded vs another party). The seller and buyer of a future transact at some specified price (if option, strike price).

\end{itemize}

\begin{definition}{Co-location}
.Having servers located at various data centers run by an exchange; saves nanoseconds when the market maker is sending and amending quites and orders to the exchange's matching-engine.
\end{definition}

\begin{remark}{OTC Trades}
.OTC (Over the Counter) trades occur off-floor, directly between two parties. These trades have significantly more counterparty risk then exchange listed contracts (no guarantee that the other party/firm will honor their contract or has sufficient funds).
\end{remark}

In contrast, for this to occur when trading on an exchange, the opposing entity, its clearing company, AND the exchange would need to default on the trade.

Options on futures are regulated and governed by the Commodity Futures Trading Commission (CFTC) while options on equities are regulated by the Securities and Exchange Commission (SEC).

There are two main exhanges upon which options on futures are traded:

\begin{itemize}
  \item CME Group: COMEX (metals), NYMEX (energy), CBOT (grains, treasures), CME (currencies, Eurodollars, SP500, livestock),
  \item ICE: Brent \& WTI Oil, sugar, coffee, cocoa, OJ, Russell 2000 index, USD index

\end{itemize}

\subsection{Forward Contract}

Used to hedge unpredictability in profits. Two parties agree ahead of time to transact at a specified price (Apple farmer and pie chain). This prevents vol due to unexpected prices. Formally, an obligation to transact in the future at a specified price. (producers and consumers)

The exchange stands to profit off these futures contracts (by taking on the counterparty risk) \textrightarrow essentially playing the middleman by creating a contract with other parties and pocketing the edge. Requires other parties to set aside margins to further protect against losses.

\section{Options}

\begin{definition}{Call}
.A call option is the right, but not the oblifation to buy the asset for a specified price (strike price) on or before a specified expiration date.
\end{definition}

\begin{definition}{Put}
.A put option is the right, but not the oblifation to sell the asset for a specified price (strike price) on or before a specified expiration date.
\end{definition}

Note that `theo' is a subjective theoretical value calculated by the specific market maker, determining whether or not they will participate in selling/buying contracts at this specific strike price.

\begin{definition}{Turnover Quantity}
.This is the total contracts traded in the current day.
\end{definition}

Bid offset is the required edhe around the `theo' value in order to auto-trade the option. Note that bid offset = ask offset.

Important greeks include spline delta and vega.

Tick sizes and multipliers vary based on the future being traded. Generally, profit = ticks made * tick size * multiplier, where ticks made = contracts * indiv. ticks made.

Furthermore, options typically have a multipler ONTO the underlying stock/future. For stocks, it is 100 and for futures it is 1. (You must further multiply this value).

\begin{remark}{Edge vs Cash}
.Oftentimes, trades will be discussed in terms of edge in points (ticks) vs the current theo. (more applicable for options). 1.00 \textrightarrow 1.04 implies making 4 cents of edge. If ticks are in pennies, then it is also 4 ticks of edge.
\end{remark}

To translate this edge to cash, we do multiplier * contracts * edge; in prev. example with a multipier of 1000, $\text{profit} = 0.04 * 1000 * 20 = 800$ in theoretical profit (vs current theo).

\section{Electronic Markets}

Any operation requiring speed is sent through servers located close to the matching engine of an exchange. This includes IOC orders, pre-calculated values, and FPGAs. Traders interact with the server that then interacts with the matching engine; they change things that require less speed (settings, params, quotes on/off). The Clearing Firm is also connected, which holds trades, does accounting, and communicates with the Clearing House. The Clearing House matches trades between clearing firms (between all different market participants).

\end{document}